\documentclass[dvipdfmx]{jsarticle}
\usepackage{mymacros}
\usepackage{amsmath,amssymb,amsthm}
\usepackage{tcolorbox}
\usepackage{mathrsfs}
\usepackage{tikz}
\newtheorem{theo}{定理}
\newtheorem{defi}{定義}
\newtheorem{lemm}{補題}
\newtheorem{plob}{演習問題}
\newtheorem{axio}{公理}
\title{都数新入生ゼミ}
\author{石塚 伶}
\date{}
\begin{document}
\maketitle

\subsection{対称性の個数}

図のような $1/n$回転でもとに戻るときに $n$回対称であるという。

\begin{defi}
ある図形の一つの軸における回転による対称性の個数をその軸によって回転したときに図形の位置を不変とする回転角の個数とする(ただし回転角は $2\pi$より小さいものとする)
\end{defi}

\subsubsection{図形の対称性}

例えば、 $L$は $T$を $1/3$回転で不変にする。つまり $L$は1つあたり $2\pi /3$回転と $4\pi /3$回転を持つ。
よって $L$のような回転軸は全部で4つあるため $T$の $L$による対称性の個数は $2 \times 4 = 8$個ある。

また、 $M$は $T$を $1/2$回転でもとに戻すので一つの $M$あたり2回対称であり $\pi$回転を持つ。 $M$は $L$において3つあるため対称性の個数は $1 \times 3 = 3$個ある。

そして $T$自身を $2\pi$回転させるということによる対称性を考えれば正四面体 $T$は12個の対称性をもつ。


六角形や正角錐においても上のような対称性の個数を考えられる。

六角形には回転軸が、向かい合った2つの頂点を通るもの $A$が3つ、向かい合った2つの辺を通るもの $B$が3つ、中心を通るもの $C$が1つ、恒等的なものが1つある。
正角錐には一回に $\pi /6$の回転を行うものがある。

\begin{defi}
回転が可換であるとは2つの回転をどのように取ってきても回転の順序によらず結果が同じになるということである。
\end{defi}

正角錐においてはすべての場合において可換となっている。
正四面体においては可換でないものがあり、それは $L$と $M$を両方共使うときに起きる。
軸は空間に対して不動であり、図形の回転によって変動することはない。

また、同じ回転を2回行ったときに図形に恒等的な対象性を与えるものが存在し、それは $\pi$回転である。

正角錐においては $\pi$回転が唯一のものである。
正四面体においては $M$による $\pi$回転のみである。
六角形においては $A,B,C$それぞれによる $\pi$回転がある。

六角形では $A,B$を行った後に適切な $C$による $\pi$回転を行うことで恒等的な対象性を与えることもできる。
これは可換である。

\subsubsection{代数的構造}
対称性の個数だけでは回転の分類に不十分であるから互いがどのように掛け合わされるかも考えなければならない。
つまり回転操作の間にかけ合わせということを定義する必要がある。

\begin{defi}
$v$と $u$の回転をかけ合わせるとは、先に $v$を行って次に $u$を行うことを指し、これを $uv$と書くこととし、これを $v$と $u$の積と呼ぶこととする。
\end{defi}

この積は関数の合成のように後ろから先に行う。
恒等的な回転は $e$と書くことにする。
この回転は $eu = ue = u$を満たす特殊な動きをする。
この $e$を単位元と呼ぶことにする。
また任意の回転操作 $u$には逆元 $u^{-1}$が存在し、これは
$uu^{-1} = u^{-1}u = e$をみたす。
逆元は回転操作において逆方向に回すことと同値である。
ただし、回転は一方向に向けて定義されているので逆方向を表すためにはその回転を繰り返すことにより表す。
また足し算や掛け算の結合法則もなりたつ。
つまり $u(vw) = (uv)w$が満たされる。
このような代数的な構造を持つ12の対象性は正四面体の回転対称性の群を形成する。

\subsection{群の公理}

群(group)とは、集合 $G$が $G$上に積を持ち以下の3つの公理を満たすときを言う。
また、 $G$の元の組 $x,y$に関する積 $xy$が必ず $G$内に存在する。(演算が閉じている)

\begin{axio}
  積は結合的である。つまり結合法則 $(xy)z = x(yz)$が $G$からどのような3つの元を持ってきたとしても成り立つ。
\end{axio}

\begin{axio}
  単位元が存在する。つまり $xe = ex =x$となるような単位元と呼ばれる元 $e$が $G$の任意の元 $x$に対して存在する。
\end{axio}

\begin{axio}
  逆元が存在する。つまり $G$の任意の元 $x$に対して逆元と呼ばれる $xx^{-1} = x^{-1}x = e$となる $x^{-1}$が必ず存在する。
\end{axio}

群において元を取る順番は重要で $xy$と $yx$は必ずしも同じとは限らない。
積と言われるがいわゆる掛け算ではなく演算に対して2つの元から1つの元を表すもの。

\subsubsection{Lorentz群}
ローレンツ(Lorentz)群とは下記のような形の行列の元からなる。

\[
\begin{pmatrix}
    \cosh u & \sinh u \\
    \sinh u & \cosh u \\
\end{pmatrix}
\]

$ \cosh \sinh $ は $x = \cosh u , y = \sinh u $が双曲線 $x^2 - y^2 = 1$を決定するため双曲線関数と言われていてエクスポネンシャルを用いることで下記のようにも書ける。
\[
  \sinh u = \frac{e^u - e^{-u}}{2} \, , \, \cosh u = \frac{e^u + e^{-u}}{2}
\]

双曲線関数は下記のような式を満たす。
\begin{eqnarray*}
  \cosh (u \pm v) & = & \cosh u \cosh v \pm \sinh u \sinh v \\
  \sinh (u \pm v) & = & \sinh u \cosh v \pm \cosh u \sinh v \\
\end{eqnarray*}
この式より下のような行列の積の計算となり左辺と右辺ともに行列の集合に入っている。
\[
\begin{pmatrix}
  \cosh u & \sinh u \\
  \sinh u & \cosh u \\
\end{pmatrix}
\begin{pmatrix}
  \cosh v & \sinh v \\
  \sinh v & \cosh v \\
\end{pmatrix}
=
\begin{pmatrix}
  \cosh (u + v) & \sinh (u + v) \\
  \sinh (u + v) & \cosh (u + v) \\
\end{pmatrix}
\]
また $2 \times 2$の正方行列の単位行列は $\sinh \cosh$の定義から下記のようになりこれも行列の集合に入っている。
\[
\begin{pmatrix}
  1 & 0 \\
  0 & 1
\end{pmatrix}
=
\begin{pmatrix}
  \cosh 0 & \sinh 0 \\
  \sinh 0 & \cosh 0 \\
\end{pmatrix}
\]
逆元としては上の二式から下記のように取れば良い。
\[
\begin{pmatrix}
  \cosh (-u) & \sinh (-u) \\
  \sinh (-u) & \sinh (-u) \\
\end{pmatrix}
\]

\subsubsection{群の公理より成り立つ事柄}
\begin{theo}
  一つの群に単位元は唯一つしか存在しない。
\end{theo}

群 $G$に単位元 $e,e'$が存在するとする。
このとき $e$が単位元であることからから $ee' = e'$が成り立つ。
また $e'$が単位元であるから $e'e = e$が成り立つ。
したがって $e' = ee' = e$より単位元は唯一つとなる。

\begin{theo}
  一つの元 $x$に対して逆元は唯一つのみ存在する。
\end{theo}

群 $G$の元 $x$の逆元が $y,z$の2つ存在するとする。
このとき $G$が群であることより単位元 $e$から次のようになる。
\begin{eqnarray*}
  y & = & ey    eが単位元であるから e = zxより \\
    & = & (zx)y   結合法則から \\
    & = & z(xy)   yが逆元であるから \\
    & = & ze    eは単位元なので \\
    & = & z
\end{eqnarray*}
より $y = z$となるから逆元は一つの元に対してただ一つのみ定まる。

\end{document}
